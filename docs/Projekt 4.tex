\documentclass[a4paper, 12pt]{article}
\usepackage[utf8]{inputenc}
\usepackage[T1]{fontenc}
\usepackage[polish]{babel}
\usepackage{tikz}
\usepackage{amsmath}
\usepackage{listings}
\usetikzlibrary{shapes, arrows, positioning, fit, backgrounds}

\tikzstyle{rect} = [draw, rectangle, rounded corners, fill=white, minimum width=1.75cm, minimum height=0.7cm]
\tikzstyle{line} = [draw, -latex]

\title{[PROI] Projekt 4.}
\author{Adrian Brodzik}

\begin{document}

\maketitle

\section*{Zadanie}
Zaprojektować prosty edytor graficzny, w którym można rysować obiekty proste (np. kółko) oraz złożone (zbiór obiektów). Obrazy generować w formacie \textit{Scalable Vector Graphics (SVG)}. Zachować odpowiedni stopień abstrakcji i możliwość rozwoju aplikacji, w tym dodawanie kolejnych formatów plików.

\section*{Rozwiązanie}
Niech \texttt{Attribute} będzie prostą strukturą danych postaci \texttt{(klucz, wartość)}. Niech \texttt{Object} będzie zbiorem danych typu \texttt{Attribute}. Wtedy \texttt{Object} może reprezentować figurę o pewnych cechach typu \texttt{Attribute}. Taka separacja klas pozwala figurom posiadać dowolną liczbę cech oraz ułatwia generowanie plików zawierających te figury.
\\\\
Niech \texttt{Double} będzie prostą strukturą danych zawierającą liczbę rzeczywistą oraz wartość logiczną. Niech \texttt{AttributeDouble} dziedziczy \texttt{Attribute} i przechowywał wartości typu \texttt{Double}. Wtedy możemy przechowywać cechy będącymi współrzędnymi absolutnymi lub relatywnymi (tzn. skalowalnymi).
\\\\
Niech \texttt{Complex} dziedziczy \texttt{Object} i będzie zbiorem danych typu \texttt{Object}. Wtedy \texttt{Complex} pozwala tworzyć złożone figury i obrazki.

\section*{Testowanie}
Testy przeprowadzono za pomocą biblioteki \texttt{doctest} i serwisu ciągłej integracji \texttt{Travis CI}. Testowano konstruowanie figur prostych i złożonych, dodawanie i modyfikacja atrybutów oraz generowanie obiektów o relatywnych wymiarach i współrzędnych.

\section*{Przykład użycia}
\begin{lstlisting}[language=C++]
#include "simplegraphics/simplegraphics.hpp"

namespace SG = SimpleGraphics;

int main()
{
    SG::Complex img;

    img.add(SG::Rectangle(0, 0, 100, 50, "#FF0000"));
    img.add(SG::Rectangle(50, 50, 50, "#00FF00"));
    img.add(SG::Circle(25, 75, 25, "#0000FF"));

    img.save("test.svg");

    return 0;
}
\end{lstlisting}

\section*{Hierarchia klas}
\subsection*{Object}
Figury proste i złożone.
\begin{center}
    \begin{tikzpicture}
    \node [rect] (object) {Object};
    
    \coordinate [below=0.5cm of object] (c1);
    
    \node [rect, below=0.5cm of c1] (ellipse) {Ellipse};
    \node [rect, left=0.5cm of ellipse] (complex) {Complex};
    \node [rect, left=0.5cm of complex] (circle) {Circle};
    \node [rect, right=0.5cm of ellipse] (line) {Line};
    \node [rect, right=0.5cm of line] (rect) {Rectangle};
    
    \path [line] (object) -- (c1) -- (ellipse);
    \path [line] (object) -- (c1) -| (complex);
    \path [line] (object) -- (c1) -| (circle);
    \path [line] (object) -- (c1) -| (line);
    \path [line] (object) -- (c1) -| (rect);
    \end{tikzpicture}
\end{center}

\subsection*{Attribute}
Cechy figur prostych i złożonych.
\begin{center}
    \begin{tikzpicture}
    \node [rect] (a1) {Attribute};
    
    \coordinate [below=0.5cm of a1] (c1);
    \coordinate [left=2cm of c1] (c2);
    \coordinate [right=2cm of c1] (c3);
    
    \node [rect, below=0.5cm of c2] (a2) {AttributeDouble};
    \node [rect, below=0.5cm of c3] (a3) {AttributeString};
    
    \path [line] (a1) -- (c1) -| (c2) -| (a2);
    \path [line] (a1) -- (c1) -| (c3) -| (a3);
    \end{tikzpicture}
\end{center}

\subsection*{Double}
Liczba rzeczywista, wymiar, współrzędna, która może być relatywna.
\begin{center}
    \begin{tikzpicture}
    \node [rect] (double) {Double};
    \end{tikzpicture}
\end{center}

\subsection*{Point}
Współrzędne punktu, para $(x,y)$ typu \texttt{Double}.
\begin{center}
    \begin{tikzpicture}
    \node [rect] (point) {Point};
    \end{tikzpicture}
\end{center}

\end{document}