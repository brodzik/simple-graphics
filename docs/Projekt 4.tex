\documentclass[a4paper, 12pt]{article}
\usepackage[T1]{fontenc}

\title{[PROI] Projekt 4.}
\author{Adrian Brodzik}

\begin{document}

\maketitle

\section*{Zadanie}
Zaprojektować prosty edytor graficzny, w którym można rysować obiekty proste (np. kółko) oraz złożone (zbiór obiektów). Obrazy generować w formacie \textit{Scalable Vector Graphics (SVG)}. Zachować odpowiedni stopień abstrakcji i możliwość rozwoju aplikacji, w tym dodawanie kolejnych formatów plików.

\section*{Rozwiązanie}
Niech \texttt{Attribute} będzie prostą strukturą danych postaci \texttt{(klucz, wartość)}. Niech \texttt{Object} będzie zbiorem danych typu \texttt{Attribute}. Wtedy \texttt{Object} może reprezentować figurę o pewnych cechach typu \texttt{Attribute}. Taka separacja klas pozwala figurom posiadać dowolną liczbę cech oraz ułatwia generowanie plików zawierających te figury.
\\\\
Niech \texttt{Double} będzie prostą strukturą danych zawierającą liczbę rzeczywistą oraz wartość logiczną. Niech \texttt{AttributeDouble} dziedziczy \texttt{Attribute} i przechowywał wartości typu \texttt{Double}. Wtedy możemy przechowywać cechy będącymi współrzędnymi absolutnymi lub relatywnymi (tzn. skalowalnymi).
\\\\
Niech \texttt{Complex} dziedziczy \texttt{Object} i będzie zbiorem danych typu \texttt{Object}. Wtedy \texttt{Complex} pozwala tworzyć złożone figury i obrazki.

\pagebreak

\section*{Hierarchia klas}
\begin{enumerate}
    \item \texttt{Attribute} - klasa bazowa atrybutów
    \begin{enumerate}
        \item \texttt{AttributeDouble} - atrybut liczby rzeczywistej (absolutnej/relatywnej)
        \item \texttt{AttributeString} - atrybut tablicy znaków
    \end{enumerate}
    \item \texttt{Double} - liczba rzeczywista 
    \item \texttt{Point} - punkt (współrzędne x, y)
    \item \texttt{Object} - zbiór atrybutów
    \begin{enumerate}
        \item \texttt{Circle} - kółko
        \item \texttt{Complex} - zbiór obiektów
        \item \texttt{Ellipse} - elipsa
        \item \texttt{Line} - linia
        \item \texttt{Rectangle} - prostokąt
    \end{enumerate}
\end{enumerate}

\end{document}